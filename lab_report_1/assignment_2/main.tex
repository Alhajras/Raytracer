%%%%%%%%%%%%%%%%%%%%%%%%%%%%%%%%%%%%%%%%%
% Lachaise Assignment
% LaTeX Template
% Version 1.0 (26/6/2018)
%
% This template originates from:
% http://www.LaTeXTemplates.com
%
% Authors:
% Marion Lachaise & François Févotte
% Vel (vel@LaTeXTemplates.com)
%
% License:
% CC BY-NC-SA 3.0 (http://creativecommons.org/licenses/by-nc-sa/3.0/)
% 
%%%%%%%%%%%%%%%%%%%%%%%%%%%%%%%%%%%%%%%%%

%----------------------------------------------------------------------------------------
%	PACKAGES AND OTHER DOCUMENT CONFIGURATIONS
%----------------------------------------------------------------------------------------

\documentclass{article}


%%%%%%%%%%%%%%%%%%%%%%%%%%%%%%%%%%%%%%%%%
% Lachaise Assignment
% Structure Specification File
% Version 1.0 (26/6/2018)
%
% This template originates from:
% http://www.LaTeXTemplates.com
%
% Authors:
% Marion Lachaise & François Févotte
% Vel (vel@LaTeXTemplates.com)
%
% License:
% CC BY-NC-SA 3.0 (http://creativecommons.org/licenses/by-nc-sa/3.0/)
% 
%%%%%%%%%%%%%%%%%%%%%%%%%%%%%%%%%%%%%%%%%

%----------------------------------------------------------------------------------------
%	PACKAGES AND OTHER DOCUMENT CONFIGURATIONS
%----------------------------------------------------------------------------------------

\usepackage{amsmath,amsfonts,stmaryrd,amssymb} % Math packages

\usepackage{enumerate} % Custom item numbers for enumerations

\usepackage[ruled]{algorithm2e} % Algorithms

\usepackage[framemethod=tikz]{mdframed} % Allows defining custom boxed/framed environments

\usepackage{listings} % File listings, with syntax highlighting
\lstset{
	basicstyle=\ttfamily, % Typeset listings in monospace font
}

%----------------------------------------------------------------------------------------
%	DOCUMENT MARGINS
%----------------------------------------------------------------------------------------

\usepackage{geometry} % Required for adjusting page dimensions and margins

\geometry{
	paper=a4paper, % Paper size, change to letterpaper for US letter size
	top=2.5cm, % Top margin
	bottom=3cm, % Bottom margin
	left=2.5cm, % Left margin
	right=2.5cm, % Right margin
	headheight=14pt, % Header height
	footskip=1.5cm, % Space from the bottom margin to the baseline of the footer
	headsep=1.2cm, % Space from the top margin to the baseline of the header
	%showframe, % Uncomment to show how the type block is set on the page
}

%----------------------------------------------------------------------------------------
%	FONTS
%----------------------------------------------------------------------------------------

\usepackage[utf8]{inputenc} % Required for inputting international characters
\usepackage[T1]{fontenc} % Output font encoding for international characters

\usepackage{XCharter} % Use the XCharter fonts

%----------------------------------------------------------------------------------------
%	COMMAND LINE ENVIRONMENT
%----------------------------------------------------------------------------------------

% Usage:
% \begin{commandline}
%	\begin{verbatim}
%		$ ls
%		
%		Applications	Desktop	...
%	\end{verbatim}
% \end{commandline}

\mdfdefinestyle{commandline}{
	leftmargin=10pt,
	rightmargin=10pt,
	innerleftmargin=15pt,
	middlelinecolor=black!50!white,
	middlelinewidth=2pt,
	frametitlerule=false,
	backgroundcolor=black!5!white,
	frametitle={Command Line},
	frametitlefont={\normalfont\sffamily\color{white}\hspace{-1em}},
	frametitlebackgroundcolor=black!50!white,
	nobreak,
}

% Define a custom environment for command-line snapshots
\newenvironment{commandline}{
	\medskip
	\begin{mdframed}[style=commandline]
}{
	\end{mdframed}
	\medskip
}

%----------------------------------------------------------------------------------------
%	FILE CONTENTS ENVIRONMENT
%----------------------------------------------------------------------------------------

% Usage:
% \begin{file}[optional filename, defaults to "File"]
%	File contents, for example, with a listings environment
% \end{file}

\mdfdefinestyle{file}{
	innertopmargin=1.6\baselineskip,
	innerbottommargin=0.8\baselineskip,
	topline=false, bottomline=false,
	leftline=false, rightline=false,
	leftmargin=2cm,
	rightmargin=2cm,
	singleextra={%
		\draw[fill=black!10!white](P)++(0,-1.2em)rectangle(P-|O);
		\node[anchor=north west]
		at(P-|O){\ttfamily\mdfilename};
		%
		\def\l{3em}
		\draw(O-|P)++(-\l,0)--++(\l,\l)--(P)--(P-|O)--(O)--cycle;
		\draw(O-|P)++(-\l,0)--++(0,\l)--++(\l,0);
	},
	nobreak,
}

% Define a custom environment for file contents
\newenvironment{file}[1][File]{ % Set the default filename to "File"
	\medskip
	\newcommand{\mdfilename}{#1}
	\begin{mdframed}[style=file]
}{
	\end{mdframed}
	\medskip
}

%----------------------------------------------------------------------------------------
%	NUMBERED QUESTIONS ENVIRONMENT
%----------------------------------------------------------------------------------------

% Usage:
% \begin{question}[optional title]
%	Question contents
% \end{question}

\mdfdefinestyle{question}{
	innertopmargin=1.2\baselineskip,
	innerbottommargin=0.8\baselineskip,
	roundcorner=5pt,
	nobreak,
	singleextra={%
		\draw(P-|O)node[xshift=1em,anchor=west,fill=white,draw,rounded corners=5pt]{%
		Question \theQuestion\questionTitle};
	},
}

\newcounter{Question} % Stores the current question number that gets iterated with each new question

% Define a custom environment for numbered questions
\newenvironment{question}[1][\unskip]{
	\bigskip
	\stepcounter{Question}
	\newcommand{\questionTitle}{~#1}
	\begin{mdframed}[style=question]
}{
	\end{mdframed}
	\medskip
}

%----------------------------------------------------------------------------------------
%	WARNING TEXT ENVIRONMENT
%----------------------------------------------------------------------------------------

% Usage:
% \begin{warn}[optional title, defaults to "Warning:"]
%	Contents
% \end{warn}

\mdfdefinestyle{warning}{
	topline=false, bottomline=false,
	leftline=false, rightline=false,
	nobreak,
	singleextra={%
		\draw(P-|O)++(-0.5em,0)node(tmp1){};
		\draw(P-|O)++(0.5em,0)node(tmp2){};
		\fill[black,rotate around={45:(P-|O)}](tmp1)rectangle(tmp2);
		\node at(P-|O){\color{white}\scriptsize\bf !};
		\draw[very thick](P-|O)++(0,-1em)--(O);%--(O-|P);
	}
}

% Define a custom environment for warning text
\newenvironment{warn}[1][Warning:]{ % Set the default warning to "Warning:"
	\medskip
	\begin{mdframed}[style=warning]
		\noindent{\textbf{#1}}
}{
	\end{mdframed}
}

%----------------------------------------------------------------------------------------
%	INFORMATION ENVIRONMENT
%----------------------------------------------------------------------------------------

% Usage:
% \begin{info}[optional title, defaults to "Info:"]
% 	contents
% 	\end{info}

\mdfdefinestyle{info}{%
	topline=false, bottomline=false,
	leftline=false, rightline=false,
	nobreak,
	singleextra={%
		\fill[black](P-|O)circle[radius=0.4em];
		\node at(P-|O){\color{white}\scriptsize\bf i};
		\draw[very thick](P-|O)++(0,-0.8em)--(O);%--(O-|P);
	}
}

% Define a custom environment for information
\newenvironment{info}[1][Info:]{ % Set the default title to "Info:"
	\medskip
	\begin{mdframed}[style=info]
		\noindent{\textbf{#1}}
}{
	\end{mdframed}
}
 % Include the file specifying the document structure and custom commands

%----------------------------------------------------------------------------------------
%	ASSIGNMENT INFORMATION
%----------------------------------------------------------------------------------------

\title{Information Retrieval: Assignment \#3} % Title of the assignment

\author{Alhajssssras Algdairy\\ \texttt{alhajras.algdairy@gmail.com, 4963555, aa382}} % Author name and email address

\date{University of Freiburg --- \today} % University, school and/or department name(s) and a date

%----------------------------------------------------------------------------------------

\begin{document}

\maketitle % Print the title

%----------------------------------------------------------------------------------------
%	INTRODUCTION
%----------------------------------------------------------------------------------------

\section*{Exercise 1} % Unnumbered section

In order to prove this we need to make an example. We will introduce two random lists $A$ and $B$, where $A$ has $k$ elements and $B$ has $n$ elements, where $ k \leq n $

\begin{equation}
	A = \{1, 2, 4, 8\}
\end{equation}
\begin{equation}
	B = \{0, \underline{1}, \underline{2}
	, 3, \underline{4}, 5, 6, 7, \underline{8}, 9, 10, 11, 12, 13\}
\end{equation}

By applying the galloping-search algorithm to allocate $A[i]$ inside $B$, we note the follow.\vspace{2mm} %5mm vertical space
\linebreak
When $di = 1$, we get 1.$O(1) = 1$, therefore $log(1) = 0$ \linebreak
When $di = 2$, we get $2.O(1) = 2$, therefore $log(2) = 1$ \linebreak
When $di = 4$, we get $3.O(1) = 3$, therefore $log(4) = 2$ \linebreak
When $di = 8$, we get $4.O(1) = 4$, therefore $log(8) = 3$ \linebreak

As we can see there is a direct relationship between the complexity and the $log$ of the $di$, therefore the big O, has the next relationship for each step of the comparison $O(di) = log(di) + 1$, this is only for one step, if we want to sum it up for each element in $A$, we get the follow. 

\begin{equation}
	O(k + \sum_{i=1}^{k} log(di))
\end{equation}

\clearpage

%----------------------------------------------------------------------------------------
%	PROBLEM 1
%----------------------------------------------------------------------------------------

\section*{Exercise 2} % Unnumbered section

We first introduce the Lagrange multipliers equation. 

\begin{equation}
	L(x_1, …, x_n, \lambda) := f(x_1, …, x_n) - \lambda  g(x_1, …, x_n)
\end{equation}

In this problem our input functions are: 

\begin{equation}
	Maximize = \sum_{i=1}^{k} log(di)
\end{equation}

\begin{equation}
	Constraint = \sum_{i=1}^{k}di \leq n
\end{equation}

By substituting both equations 5 and 6 in 4 we get: 
\begin{flushleft}
\begin{equation}
	\begin{aligned}
	 L = \sum_{i=1}^{k} log(di) - \lambda (\sum_{i=1}^{k}di - n) \\
	 L = (log(d_1) + log(d_2) .. log(d_k)) - \lambda ((d_1 + d_2 .. d_k) - n) \\ 
	 \frac{\partial L}{\partial d_1} = \frac{1}{d_1} - \lambda = 0, \lambda = \frac{1}{d_1} \\
	 \frac{\partial L}{\partial d_2} = \frac{1}{d_2} - \lambda = 0, \lambda = \frac{1}{d_2} \\
	 \frac{\partial L}{\partial d_k} = \frac{1}{d_k} - \lambda = 0, \lambda = \frac{1}{d_k} \\
	 L = \sum_{i=1}^{k} log(di) - \lambda (\sum_{i=1}^{k}di - n) \\
	\frac{1}{d_1} = \frac{1}{d_2} = \frac{1}{d_k} \\
	d_k = \frac{n}{x}
	\end{aligned}
\end{equation}
\end{flushleft}
\begin{flushleft}
The only condition where it makes the previous terms equal independent of $x$, is when all terms  $d_i = \frac{n}{x}$ are evenly equal. This can be done when we take the list $B$ and divide it into evenly steps where all $d_i$ are equals. 
\end{flushleft}

\begin{flushleft}
Therefore, we get $d_i = \frac{n}{k}$, to distribute the smaller list inside the bigger list symmetrically.
\end{flushleft}
From the Exercise 1, we have. 

\begin{flushleft}
	\begin{equation}
		\begin{aligned}
	O(k + \sum_{i=1}^{k} log(di)) \\
	@d_i = \frac{n}{k}, O(k + k.log( \frac{n}{k})) \\
	O(k(1 + log( \frac{n}{k})))
	\end{aligned}
\end{equation}
\end{flushleft}
\begin{flushleft}
The simpler bounds is not correct, when we think of an edge case, such as when the $d_i = 1$ for all elements, this means we get hit after each jump in the algorithm, without exponentially increasing. This case gives $k$ number of jumps, however, when we use the algorithm with out the constant $k$ we get: 
\end{flushleft}

$  \sum_{i=1}^{k} log(d_i) = \sum_{i=1}^{k} log(1) = 0$

\begin{flushleft}
Hence adding the $k$ fix this issue.
\end{flushleft}

\clearpage



%------------------------------------------------

\section*{Exercise 3} % Unnumbered section

In order to prove this we need to illustrate the next example. Considering a list with $n = 100$ elements and a smaller list with only $k = 2$, we know that $d_i = \frac{100}{2} = 50$, now with such a big deviation between the two elements. The Zipper search algorithm will start searching for the first element which will be found in the middle of the list at $B[50]$ this will take $50$ comparisons which is eqaul to $\frac{n}{2}$, same procedure will be done to the second element, it will take the same steps, therefore the number of comparisons we have in total $\frac{n}{2} + \frac{n}{2} = n$, therefore it will have always O(n) which is a linear time. This case can be noticed for any arbitrary list as long as we have the $d_i$ evenly distributed. 

Considering the galloping search when we look at its complexity which is logarithmic and non linear as the Zipper in this case. This can be seen in $equation 8$. Therefore $O(log(n))$ is always better then $O(n)$

\end{document}
